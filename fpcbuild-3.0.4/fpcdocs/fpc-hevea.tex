%
%   $Id: fpc-hevea.tex,v 1.1 2003/03/16 15:22:18 peter Exp $
%
%   This file is part of the FPC documentation.
%   Copyright (C) 1997, by Michael Van Canneyt
%
%   The FPC documentation is free text; you can redistribute it and/or
%   modify it under the terms of the GNU Library General Public License as
%   published by the Free Software Foundation; either version 2 of the
%   License, or (at your option) any later version.
%
%   The FPC Documentation is distributed in the hope that it will be useful,
%   but WITHOUT ANY WARRANTY; without even the implied warranty of
%   MERCHANTABILITY or FITNESS FOR A PARTICULAR PURPOSE.  See the GNU
%   Library General Public License for more details.
%
%   You should have received a copy of the GNU Library General Public
%   License along with the FPC documentation; see the file COPYING.LIB.  If not,
%   write to the Free Software Foundation, Inc., 59 Temple Place - Suite 330,
%   Boston, MA 02111-1307, USA.

\usepackage{makeidx}
\usepackage{ifthen}

%
% Fake packages
%
\newcommand{\xspace}{ }
\newenvironment{multicols}[1]{}{}

\newcommand{\docdescription}[1]{\def\@FPCDescription{#1}}
\gdef\@FPCDescription{}%
\newcommand{\docversion}[1]{\def\@FPCVersion{#1}}
\gdef\@FPCVersion{}%

%
% FPC environments
%
% Remarks
\newenvironment{remark}{\par\textbf{Remark:} }{\par}
% List
\newenvironment{FPCList}{\begin{list}}{\end{list}}
% For Tables.
\newenvironment{FPCtable}[2]{\begin{table}\caption{#2}\begin{center}\begin{tabular}{#1}}{\end{tabular}\end{center}\end{table}}
% The same, but with label in third argument (tab:#3)
\newenvironment{FPCltable}[3]{\begin{table}\caption{#2}\label{tab:#3}\begin{center}\begin{tabular}{#1}}{\end{tabular}\end{center}\end{table}}

%
% Refs
%
\newcommand{\seefig}[1]{Figure \ref{fig:#1}\xspace}
\newcommand{\seefl}[2]{\ahref{fu:#2}{#1}}
\newcommand{\seepl}[2]{\ahref{pro:#2}{#1}}
\newcommand{\seetypel}[2]{\ahref{ty:#2}{#1}}
\newcommand{\seeconstl}[2]{\ahref{co:#2}{#1}}
\newcommand{\seevarl}[2]{\ahref{var:#2}{#1}}
\newcommand{\seec}[1]{chapter \ref{ch:#1}\xspace}
\newcommand{\sees}[1]{section \ref{se:#1}\xspace}
\newcommand{\seeo}[1]{\var{-#1}, (see \ref{option:#1})\xspace}
\newcommand{\seet}[1]{table (\ref{tab:#1})\xspace}

%
% Function/procedure environments
%
\newenvironment{functionl}[2]{\subsection{#1}\index{#1}\label{fu:#2}\begin{FPCList}}{\end{FPCList}}
\newenvironment{procedurel}[2]{\subsection{#1}\index{#1}\label{pro:#2}\begin{FPCList}}{\end{FPCList}}
\newenvironment{method}[2]{\subsection{#1}\index{#1}\label{#2}\begin{FPCList}}{\end{FPCList}}
\newenvironment{property}[2]{\subsection{#1}\index{#1}\label{#2}\begin{FPCList}}{\end{FPCList}}
\newenvironment{function}[1]{\begin{functionl}{#1}{#1}}{\end{functionl}}
\newenvironment{procedure}[1]{\begin{procedurel}{#1}{#1}}{\end{procedurel}}
\newenvironment{typel}[2]{\subsection{#1}\index{#1}\label{ty:#2}\begin{FPCList}}{\end{FPCList}}
\newenvironment{type}[1]{\begin{typel}{#1}{#1}}{\end{typel}}
\newenvironment{constantl}[2]{\subsection{#1}\index{#1}\label{co:#2}\begin{FPCList}}{\end{FPCList}}
\newenvironment{constant}[1]{\begin{constantl}{#1}{#1}}{\end{constantl}}
\newenvironment{variablel}[2]{\subsection{#1}\index{#1}\label{var:#2}\begin{FPCList}}{\end{FPCList}}
\newenvironment{variable}[1]{\begin{variablel}{#1}{#1}}{\end{variablel}}
\newenvironment{ver2}{\par\textbf{version 2.0 only:}}{\par}
\newcommand{\Declaration}{\item[Declaration]\ttfamily}
\newcommand{\Description}{\item[Description]\rmfamily}
\newcommand{\Portability}{\item[Portability]\rmfamily}
\newcommand{\Errors}{\item[Errors]\rmfamily}
\newcommand{\Visibility}{\item[Visibility]\rmfamily}
\newcommand{\Access}{\item[Access]\rmfamily}
\newcommand{\Synopsis}{\item[Synopsis]\rmfamily}
\newcommand{\Arguments}{\item[Arguments]\rmfamily}
\newcommand{\SeeAlso}{\item[See also]\rmfamily}
%
% Ref without labels
%
\newcommand{\seef}[1]{\seefl{#1}{#1}}
\newcommand{\seep}[1]{\seepl{#1}{#1}}
\newcommand{\seetype}[1]{\seetypel{#1}{#1}}
\newcommand{\seevar}[1]{\seevarl{#1}{#1}}
\newcommand{\seeconst}[1]{\seeconstl{#1}{#1}}
%
% man page references don't need labels.
%
\newcommand{\seem}[2]{\texttt{#1} (#2) }
%
% HTML references
%
\newcommand{\seeurl}[2]{\ahref{#2}{#1}}
%
% for easy typesetting of variables.
%
\newcommand{\var}[1]{\texttt {#1}}
\newcommand{\file}[1]{\textsf {#1}}
\newcommand{\key}[1]{\textsc{#1}}
\newcommand{\menu}[1]{\textbf{"#1"}}
%
% Useful references.
%
\newcommand{\progref}{\ahref{../prog/prog.html}{Programmers guide}\xspace}
\newcommand{\refref}{\ahref{../ref/ref.html}{Reference guide}\xspace}
\newcommand{\userref}{\ahref{../user/user.html}{Users guide}\xspace}
\newcommand{\unitsref}{\ahref{../units/units.html}{Unit reference}\xspace}
%
% Commands to reference these things.
%
\newcommand{\olabel}[1]{\label{option:#1}}
%
% some OSes
%
\newcommand{\linux}{\textsc{linux}\xspace}
\newcommand{\unix}{\textsc{unix}\xspace}
\newcommand{\dos}  {\textsc{dos}\xspace}
\newcommand{\msdos}{\textsc{ms-dos}\xspace}
\newcommand{\ostwo}{\textsc{os/2}\xspace}
\newcommand{\windows}{\textsc{Windows}\xspace}
\newcommand{\windowsnt}{\textsc{Windows NT}\xspace}
\newcommand{\fpc}{Free Pascal\xspace}
\newcommand{\gnu}{\textsc{gnu}\xspace}
\newcommand{\atari}{\textsc{Atari}\xspace}
\newcommand{\amiga}{\textsc{Amiga}\xspace}
\newcommand{\solaris}{\textsc{Solaris}\xspace}
\newcommand{\qnx}{\textsc{QNX Realtime platform}\xspace}
\newcommand{\beos}{\textsc{BeOS}\xspace}
\newcommand{\palmos}{\textsc{PalmOS}\xspace}
\newcommand{\netbsd}{\textsc{NetBSD}\xspace}
\newcommand{\openbsd}{\textsc{OpenBSD}\xspace}
\newcommand{\win}{\textsc{Win32}\xspace}
\newcommand{\freebsd}{\textsc{FreeBSD}\xspace}
\newcommand{\tp}{Turbo Pascal\xspace}
\newcommand{\delphi}{Delphi}
%
% Some versions
%
\newcommand{\fpcversion}{1.0.8}
%
% For examples
%
\newcommand{\FPCexample}[1]{\begin{verbatim}\input{\exampledir/#1.pp}\end{verbatim}}
\newcommand{\Cexample}[1]{\begin{verbatim}\input{\exampledir/#1.c}\end{verbatim}}
\newcommand{\exampledir}{.}
\newcommand{\FPCexampledir}[1]{\renewcommand{\exampledir}{#1}}
%
% Picture including
%
\newcommand{\FPCpic}[3]{%
  \linebreak%
  \begin{center}
  \textbf{Figure \ref{fig:#3} #1}%
  \linebreak%
  \label{fig:#3}%
  \imgsrc{../pics/#2/#3.png}%
  \end{center}
  \linebreak%
}
%
% Categorical Function/procedure overviews
%
\newenvironment{funclist}{\begin{list}}{\end{list}}
\newcommand{\funcrefl}[3]{\item[\ahref{#2}{fu:#2} #3]}
\newcommand{\funcref}[2]{\item[\ahref{#1}{fu:#1} #2]}
\newcommand{\procrefl}[3]{\item[\ahref{#2}{pro:#2} #3]}
\newcommand{\procref}[2]{\item[\ahref{#1}{pro:#1} #2]}

%
% end of fpc-html.tex
%
